\documentclass[]{book}

%These tell TeX which packages to use.
\usepackage{array,epsfig}
\usepackage{amsmath}
\usepackage{amsfonts}
\usepackage{amssymb}
\usepackage{amsxtra}
\usepackage{amsthm}
\usepackage{mathrsfs}
\usepackage{color}
\usepackage[english]{babel}
\usepackage[utf8]{inputenc}

%Here I define some theorem styles and shortcut commands for symbols I use often
\theoremstyle{definition}
\newtheorem{defn}{Definition}
\newtheorem{thm}{Theorem}
\newtheorem{cor}{Corollary}
\newtheorem*{rmk}{Remark}
\newtheorem{lem}{Lemma}
\newtheorem*{joke}{Joke}
\newtheorem{ex}{Example}
\newtheorem*{sol}{Solution}
\newtheorem{prop}{Proposition}

\newcommand{\lra}{\longrightarrow}
\newcommand{\ra}{\rightarrow}
\newcommand{\surj}{\twoheadrightarrow}
\newcommand{\graph}{\mathrm{graph}}
\newcommand{\bb}[1]{\mathbb{#1}}
\newcommand{\Z}{\bb{Z}}
\newcommand{\Q}{\bb{Q}}
\newcommand{\R}{\bb{R}}
\newcommand{\C}{\bb{C}}
\newcommand{\N}{\bb{N}}
\newcommand{\M}{\mathbf{M}}
\newcommand{\m}{\mathbf{m}}
\newcommand{\MM}{\mathscr{M}}
\newcommand{\HH}{\mathscr{H}}
\newcommand{\Om}{\Omega}
\newcommand{\Ho}{\in\HH(\Om)}
\newcommand{\bd}{\partial}
\newcommand{\del}{\partial}
\newcommand{\bardel}{\overline\partial}
\newcommand{\textdf}[1]{\textbf{\textsf{#1}}\index{#1}}
\newcommand{\img}{\mathrm{img}}
\newcommand{\ip}[2]{\left\langle{#1},{#2}\right\rangle}
\newcommand{\inter}[1]{\mathrm{int}{#1}}
\newcommand{\exter}[1]{\mathrm{ext}{#1}}
\newcommand{\cl}[1]{\mathrm{cl}{#1}}
\newcommand{\ds}{\displaystyle}
\newcommand{\vol}{\mathrm{vol}}
\newcommand{\cnt}{\mathrm{ct}}
\newcommand{\osc}{\mathrm{osc}}
\newcommand{\LL}{\mathbf{L}}
\newcommand{\UU}{\mathbf{U}}
\newcommand{\support}{\mathrm{support}}
\newcommand{\AND}{\;\wedge\;}
\newcommand{\OR}{\;\vee\;}
\newcommand{\Oset}{\varnothing}
\newcommand{\st}{\ni}
\newcommand{\wh}{\widehat}

%Pagination stuff.
\setlength{\topmargin}{-.3 in}
\setlength{\oddsidemargin}{0in}
\setlength{\evensidemargin}{0in}
\setlength{\textheight}{9.in}
\setlength{\textwidth}{6.5in}
\setlength{\itemsep}{0.45in}
\pagestyle{empty}

\title{Comptutational Mathematics TC2020}
\author{\textbf{Assignment 01} \\* Sebastián Saldaña \\* A01570274}
\date{12.February.2021}

\begin{document}
\begin{center}
    {\huge Computational Mathematics TC2020}\\[1.5ex]
    {\large \textbf{Assignment 01}\\[1.5ex] 
    Sebastián A. Saldaña\\
    A01570274\\%You should put your name here
    12.02.21} %You should write the date here.
    \end{center}
    
    \vspace{0.2 cm}

    \section*{Preliminaries: Sets, relations and functions}
    \begin{enumerate}
        \item Calculate the result of the following operations (4 marks):
        \begin{enumerate}
            \item $\{a,b,c\} \times \{1,2,3,4\}$ \newline
            $S = \{(a,1),(a,2),(a,3),(a,4),(b,1),(b,2),(b,3),(b,4),(c,1),(c,2),(c,3),(c,4)\}$
            \item $\{a,\{b\},\{\{c\}\}\} \times \emptyset$ \newline $S = \emptyset$
            \item $\mathcal{P}(\{x : x \in \mathbb{N}, x < 4\})$ \newline
            $S = \{\emptyset,1,2,3,\{1,2\},\{1,3\},\{2,3\},\{1,2,3\}\}$
            \item $\vert\mathcal{P}(\{y : y \in \mathbb{Z},0 < y < 10\})\vert = 512$
        \end{enumerate}
        \item The following are relations from $\{1,2,3,4\}$ to $\{1,2,3,4\}$. Indicate which relations are \textbf{transitive}, \textbf{reflexive}, or \textbf{symmetric}. Also indicate which are \textbf{functions} and which one are only \textbf{relations}. In case of being functions, indicate if they are \textbf{total} or \textbf{partial} functions, and which ones are \textbf{injective}, \textbf{surjective} and \textbf{bijective} (4 marks).
        \begin{enumerate}
            \item $\{(2,2),(3,3),(1,1),(4,4)\}$ \emph{\textbf{This is a reflexive, transitive, and symmetric relation which is also a total, bijective function.}}
            \item $\{(1,1),(2,2),(3,3),(4,3)\}$ \emph{\textbf{This is a transtivie relation and a total,surjective function.}}
            \item $\{(1,1),(3,4),(2,2),(3,3)\}$ \emph{\textbf{This is only a relation}}
            \item $\{(1,1),(2,2),(3,3)\}$ \emph{\textbf{This is a reflexive, transitive, and symmetric relation which is also a partial, surjective function.}}
        \end{enumerate}
        \item Using the information from the lecture, investigate what is the \textbf{transitive closure}. Then, write its definition in your own words. Finally, find the transitive closure for each of the relations of the previous problem (2 marks)\footnote{Do not forget to cite the sources properly. Consider that they are reliable sources and, if possible, list two or three resouces; do not consider just the first definition you find.}.
        \newline A \textbf{transitive closure} is the set of individual relations between elements of a relation that encapsulates all of these elements that will make it transitive. 
        \begin{enumerate}
            \item $R^* = \{(2,2),(3,3),(1,1),(4,4)\}$
            \item $R^* = \{(1,1),(2,2),(3,3),(4,3)\}$
            \item $R^* = \{(1,1),(3,4),(2,2),(3,3),(4,3)\}$
            \item $R^* = \{(1,1),(2,2),(3,3)\}$
        \end{enumerate}
    \end{enumerate}
    \begin{thebibliography}{9}
    \bibitem{udacityvid}
    Udacity. (2015). \textit{Transitive Closure - Georgia Tech - Computability, Complexity, Theory: Algorithms}. Recovered from \texttt{https://www.youtube.com/watch?v=NM0mAmylfMg}
    \bibitem{gvsumath}
    GVSUMath. (2015). \textit{Transitive Closure}.\\* Recovered from: \texttt{https://www.youtube.com/watch?v=OO8Jfs9uZnc}
    \end{thebibliography}
\end{document}
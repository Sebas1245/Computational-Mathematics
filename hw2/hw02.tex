\documentclass[]{book}

%These tell TeX which packages to use.
\usepackage{array,epsfig}
\usepackage{amsmath}
\usepackage{amsfonts}
\usepackage{amssymb}
\usepackage{amsxtra}
\usepackage{amsthm}
\usepackage{mathrsfs}
\usepackage{color}
\usepackage[english]{babel}
\usepackage[utf8]{inputenc}

%Here I define some theorem styles and shortcut commands for symbols I use often
\theoremstyle{definition}
\newtheorem{defn}{Definition}
\newtheorem{thm}{Theorem}
\newtheorem{cor}{Corollary}
\newtheorem*{rmk}{Remark}
\newtheorem{lem}{Lemma}
\newtheorem*{joke}{Joke}
\newtheorem{ex}{Example}
\newtheorem*{sol}{Solution}
\newtheorem{prop}{Proposition}

\newcommand{\lra}{\longrightarrow}
\newcommand{\ra}{\rightarrow}
\newcommand{\surj}{\twoheadrightarrow}
\newcommand{\graph}{\mathrm{graph}}
\newcommand{\bb}[1]{\mathbb{#1}}
\newcommand{\Z}{\bb{Z}}
\newcommand{\Q}{\bb{Q}}
\newcommand{\R}{\bb{R}}
\newcommand{\C}{\bb{C}}
\newcommand{\N}{\bb{N}}
\newcommand{\M}{\mathbf{M}}
\newcommand{\m}{\mathbf{m}}
\newcommand{\MM}{\mathscr{M}}
\newcommand{\HH}{\mathscr{H}}
\newcommand{\Om}{\Omega}
\newcommand{\Ho}{\in\HH(\Om)}
\newcommand{\bd}{\partial}
\newcommand{\del}{\partial}
\newcommand{\bardel}{\overline\partial}
\newcommand{\textdf}[1]{\textbf{\textsf{#1}}\index{#1}}
\newcommand{\img}{\mathrm{img}}
\newcommand{\ip}[2]{\left\langle{#1},{#2}\right\rangle}
\newcommand{\inter}[1]{\mathrm{int}{#1}}
\newcommand{\exter}[1]{\mathrm{ext}{#1}}
\newcommand{\cl}[1]{\mathrm{cl}{#1}}
\newcommand{\ds}{\displaystyle}
\newcommand{\vol}{\mathrm{vol}}
\newcommand{\cnt}{\mathrm{ct}}
\newcommand{\osc}{\mathrm{osc}}
\newcommand{\LL}{\mathbf{L}}
\newcommand{\UU}{\mathbf{U}}
\newcommand{\support}{\mathrm{support}}
\newcommand{\AND}{\;\wedge\;}
\newcommand{\OR}{\;\vee\;}
\newcommand{\Oset}{\varnothing}
\newcommand{\st}{\ni}
\newcommand{\wh}{\widehat}

%Pagination stuff.
\setlength{\topmargin}{-.3 in}
\setlength{\oddsidemargin}{0in}
\setlength{\evensidemargin}{0in}
\setlength{\textheight}{9.in}
\setlength{\textwidth}{6.5in}
\setlength{\itemsep}{0.45in}
\pagestyle{empty}



\begin{document}

\begin{center}
{\huge Computational Mathematics TC2020 -- 01}\\[1.5ex]
{\large \textbf{Assignment 02}\\[1.5ex] 
Sebastián A. Saldaña\\
A01570274\\%You should put your name here
12.02.21} %You should write the date here.
\end{center}

\vspace{0.2 cm}

\subsection*{Preliminaries: Propositional logic}
\subsubsection*{Formalization (2 points)}
Formally express the following ideas using logic propositions. This means you must translate each idea to individual propositions and then link them using logic operators.
\begin{enumerate}
	\item In the poles the cold is really intense only if the planets revolve around the sun. \\* $p$ = The cold is really intense in the poles \\* $q$ = The planets revolve around the sun \\* $p \Leftrightarrow q$
	\item If it is true that Aristotle was born in Estagira and that he was the tutor of Alexander The Great, \underline{and in addition to the previous that if he was born in Estagira he was then Macedonian by birth}, \emph{then he was effectively Macedonian}. \newline 
	$p$ = Aristotle was born in Estagira \newline 
	$q$ = Aristotle was tutor of Alexander the Great \newline 
	$r$ = Aristotle was Macedonian by birth \newline
	$s$ = Aristotle was Macedonian \newline
	$(p \land q) \land (p \Rightarrow r)  \Rightarrow s$
\end{enumerate}

\subsubsection*{Truth tables (2 points)}
Write down the truth table of the following expression $(p \land (q \lor \neg p) ) \lor r$. Specify in columns as much as you can about how you are chaining each result. \newline
\begin{displaymath}
	\begin{array}{|c c c|c|c|c|}
		p & q & r & q \lor \lnot p & p \land (q \lor \lnot p) & (p \land (q \lor \lnot p)) \lor r\\
		\hline
		T & T & T & T & T & T\\
		T & T & F & T & T & T\\
		T & F & T & F & F & T\\
		T & F & F & F & F & F\\
		F & T & T & T & F & T\\
		F & T & F & T & F & F\\
		F & F & T & T & F & T\\
		F & F & F & T & F & F\\
	\end{array}
\end{displaymath}

\subsubsection*{Simplifying propositions (2 points)}
Simplify the following proposition $(\neg (p \land \neg q)) \land (p \lor q)$. 
You can use equivalences and properties of operations to make the simplification.
Workout the individual steps you follow here. \newline
$\lnot(p \land \lnot q) \equiv \lnot p \lor q$ \newline
$(\lnot p \lor q) \land (p \lor q) \equiv q$ \newline
This equivalence is proven by the following truth table: 
\begin{displaymath}
	\begin{array}{|c c|c|c|c|}
		p & q & \lnot p \lor q & p \lor q & (\lnot p \lor q) \land (p \lor q)\\
		\hline
		T & T & T & T & T\\
		T & F & F & T & F\\
		F & T & T & T & T\\
		F & F & T & F & F\\
	\end{array}
\end{displaymath}



\subsubsection*{Proof of equivalences (4 points)}
Use logic equivalences and the properties of propositions to demonstrate that the following equivalence is true: $(p \implies q) \lor (p \implies r) \equiv p \implies (q \lor r)$. Workout the individual steps you follow to demonstrate this.

You can think of this as selecting one of the two propositions and somehow modify it incrementally until it becomes the same as the other proposition.\newline
\begin{center}
	\emph{Modifying the L.H.S as follows:}
\end{center}
\begin{gather}
	p \Rightarrow q \equiv \lnot p \lor q\\
	p \Rightarrow r \equiv \lnot p \lor r\\
	(p \implies q) \lor (p \implies r) \equiv \lnot p \lor q \lor \lnot p \lor r\\
	\lnot p \lor q \lor \lnot p \lor r \equiv \lnot p \lor \lnot p \lor q \lor r \equiv \lnot p \lor q \lor r\\
	\lnot p \lor q \lor r \equiv \lnot p \lor (q \lor r) \equiv p \implies (q \lor r) 
\end{gather}
\begin{center}
	\emph{Shows we can get on the L.H.S what is on the R.H.S.} $\blacksquare$
\end{center}
\end{document}


